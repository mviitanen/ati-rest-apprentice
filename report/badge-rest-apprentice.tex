%%%%%%%%%%%%%%%%%%%%%%%%%%%%%%%%%%%%%%%%%
% Report about achieving the Java Apprentice Badge
% Using template:
% Stylish Article
% LaTeX Template
% Version 2.0 (13/4/14)
%
% This template has been downloaded from:
% http://www.LaTeXTemplates.com
%
% Original author:
% Mathias Legrand (legrand.mathias@gmail.com)
%
% License:
% CC BY-NC-SA 3.0 (http://creativecommons.org/licenses/by-nc-sa/3.0/)
%
%%%%%%%%%%%%%%%%%%%%%%%%%%%%%%%%%%%%%%%%%

%----------------------------------------------------------------------------------------
%	PACKAGES AND OTHER DOCUMENT CONFIGURATIONS
%----------------------------------------------------------------------------------------



%
%\texttt{.jar}
%In listing \ref{lst:libraryClass} 

%\begin{lstlisting}[language=Java, label=lst:libraryClass, caption=Library Class]
%
%\end{lstlisting}
%\vspace{2em}

%See the source code for \texttt{LibraryClass.java} in Appendix O on page \pageref{App:AppendixO}.


% create labels for figures, listings and tables
% TODO: format all Java keywords with \texttt{put keyword here}
% TODO: Format and clean up code
% TODO: make sure all classnames are in proper case in the text
% TODO: make sure the code is the latest
% TODO: refer to the line numbers in the code, when explkaining it
% crossreference code (if you call toString, mention it is in listing X)

\documentclass[twoside,fleqn,10pt]{SelfArx} % Document font size and equations flushed left
\usepackage{perpage} %the perpage package
\MakePerPage{footnote} %the perpage package command
\usepackage{hyphenat}
\usepackage{listings}
\usepackage[utf8]{inputenc}
\usepackage[T1]{fontenc}
\usepackage{tabulary}
\usepackage{lipsum} % Required to insert dummy text. To be removed otherwise
\usepackage[toc,page]{appendix}
\usepackage{float}
\usepackage{wrapfig}
\usepackage{pgfplots}
\usepackage[export]{adjustbox}
\usepackage{framed}
\usepgfplotslibrary{patchplots}
\usetikzlibrary{pgfplots.groupplots}
\pgfplotsset{compat=1.12} 
 \usepackage{pifont}
 \usepackage{adjustbox}
\usepackage{array}
\usepackage{booktabs}
\newcolumntype{R}[2]{%
    >{\adjustbox{angle=#1,lap=\width-(#2)}\bgroup}%
    l%
    <{\egroup}%
}
\newcommand*\rot{\multicolumn{1}{R{90}{1em}}}% no optional argument here, please!

\renewcommand*{\thefootnote}{\fnsymbol{footnote}}

%----------------------------------------------------------------------------------------
%	COLUMNS
%----------------------------------------------------------------------------------------

\setlength{\columnsep}{0.55cm} % Distance between the two columns of text
\setlength{\fboxrule}{0.75pt} % Width of the border around the abstract

%----------------------------------------------------------------------------------------
%	COLORS
%----------------------------------------------------------------------------------------

\definecolor{color1}{RGB}{0,0,90} % Color of the article title and sections
\definecolor{color2}{RGB}{0,20,20} % Color of the boxes behind the abstract and headings

\definecolor{mygreen}{rgb}{0,0.6,0}
\definecolor{mygray}{rgb}{0.5,0.5,0.5}
\definecolor{mymauve}{rgb}{0.58,0,0.82}
\definecolor{lightgray}{rgb}{0.95,0.95,0.95}
\usepackage{qrcode}	
%----------------
% LISTINGS
%----------------
\usepackage{multirow}

\lstset{ %
  prebreak=\raisebox{0ex}[0ex][0ex]
        {\ensuremath{\hookleftarrow}},
  backgroundcolor=\color{lightgray},   % choose the background color; you must add \usepackage{color} or \usepackage{xcolor}
  basicstyle=\footnotesize,        % the size of the fonts that are used for the code
  breakatwhitespace=false,         % sets if automatic breaks should only happen at whitespace
  breaklines=true,                 % sets automatic line breaking
  captionpos=b,                    % sets the caption-position to bottom
  commentstyle=\color{mygreen},    % comment style
  deletekeywords={...},            % if you want to delete keywords from the given language
  escapeinside={\%*}{*)},          % if you want to add LaTeX within your code
  extendedchars=true,              % lets you use non-ASCII characters; for 8-bits encodings only, does not work with UTF-8
  frame=single,                    % adds a frame around the code
  keepspaces=true,                 % keeps spaces in text, useful for keeping indentation of code (possibly needs columns=flexible)
  keywordstyle=\color{blue},       % keyword style
  morekeywords={*,...},            % if you want to add more keywords to the set
  numbers=left,                    % where to put the line-numbers; possible values are (none, left, right)
  numbersep=5pt,                   % how far the line-numbers are from the code
  numberstyle=\tiny\color{mygray}, % the style that is used for the line-numbers
  rulecolor=\color{black},         % if not set, the frame-color may be changed on line-breaks within not-black text (e.g. comments (green here))
  showspaces=false,                % show spaces everywhere adding particular underscores; it overrides 'showstringspaces'
  showstringspaces=false,          % underline spaces within strings only
  showtabs=false,                  % show tabs within strings adding particular underscores
  stepnumber=2,                    % the step between two line-numbers. If it's 1, each line will be numbered
  stringstyle=\color{mymauve},     % string literal style
  tabsize=2,                       % sets default tabsize to 2 spaces
  title=\lstname,                   % show the filename of files included with \lstinputlisting; also try caption instead of title
  belowskip=-\baselineskip         % no empty line after
}

\usepackage{adjustbox}
%----------------------------------------------------------------------------------------
%	HYPERLINKS
%----------------------------------------------------------------------------------------

\usepackage{hyperref} % Required for hyperlinks
\hypersetup{hidelinks,colorlinks,breaklinks=true,urlcolor=color2,citecolor=color1,linkcolor=color1,bookmarksopen=false,pdftitle={Title},pdfauthor={Author}}

%----------------------------------------------------------------------------------------
%	ARTICLE INFORMATION
%----------------------------------------------------------------------------------------

\JournalInfo{Professional Development Program Series, Vol. II, No. 1, 2017} % Journal information
\Archive{\copyright{} Copyright Intellectual Reserve, 2016} % Additional notes (e.g. copyright, DOI, review/research article)

\PaperTitle{REST Apprentice Badge Report} % Article title

\Authors{Marko Viitanen\textsuperscript{1}*} % Authors
\affiliation{\textsuperscript{1}\textit{FamilySearch, Salt Lake City}} % Author affiliation

\Keywords{REST --- Professional Development Program --- Badges} % Keywords - if you don't want any simply remove all the text between the curly brackets
\newcommand{\keywordname}{Keywords} % Defines the keywords heading name

%----------------------------------------------------------------------------------------
%	ABSTRACT
%----------------------------------------------------------------------------------------

\Abstract{As part of the professional development program we are provided a way to learn and demonstrate our development skills by earning badges. There are several badges, REST being one of them. This document describes learnings and answers to the requirements for earning the Apprentice REST Badge.}

%----------------------------------------------------------------------------------------

\begin{document}

\flushbottom % Makes all text pages the same height

\maketitle % Print the title and abstract box

\tableofcontents % Print the contents section

\thispagestyle{empty} % Removes page numbering from the first page

%----------------------------------------------------------------------------------------
%	ARTICLE CONTENTS
%----------------------------------------------------------------------------------------

\section*{Introduction} % The \section*{} command stops section numbering

\addcontentsline{toc}{section}{Introduction} % Adds this section to the table of contents

The Java Apprentice Badge is the first badge in the Java series. It covers intermediate concepts of Java programming. It is not a 101 course in programming, so many of the common language features are not covered. The requirements don't include installation, the Java language syntax, or basic computer science concepts such as looping and conditionals.

The high-level requirements\footnote{See confluence for full description of the requirements at \url{https://almtools.ldschurch.org/fhconfluence/display/Product/Core+Skills+-+Java+-+Apprentice} .} include:
\begin{itemize}
\item Object life cycle
\item Exceptions
\item Polymorphism
\item Collections
\item Using a library
\end{itemize}

\section{REST Architecture}

We look into REST

\subsection{Res Architecture}
\begin{center}\textit{RESTful Web Services architectures versus other architectures, such as RPC-style, SOAP, etc.}\end{center}

\subsubsection{Procedural Programming} 
yada yada yada
\section{Guiding Principles}

some description
\subsection{Constraints, Goals, Guiding Principles}

\subsubsection{Inheritance}\label{s:inheritance}

\begin{lstlisting}[language=Java]
abstract public class Animal {
 private String name;
 private Sociability sociability;
 private int numberOfLegs;
 private Tail tail;

 public Animal(String name, int numberOfLegs, 
     Sociability sociability, Tail tail) {
  // initialize fields
 }

 // getters and setters...
 
 // toString...
}
\end{lstlisting}

We marked this class \texttt{abstract}. It will prevent any instantiation of the base class. The user of these classes has to instantiate a subclass of \texttt{Animal}. A super class does not have to be declared \texttt{abstract} unless it contains \texttt{abstract} methods (more about that later). We could allow the creation of super classes. In this case we opt not to.

Appendix J on page \pageref{App:AppendixJ} has the the source of the class \texttt{Animal.java}.

A \texttt{Dog} "is-an" \texttt{Animal}. The \texttt{Cat} "is-an" \texttt{Animal}. They inherit all protected and public methods and fields of \texttt{Animal}. See table \ref{tab:accessmodifiers} on page \pageref{tab:accessmodifiers} for details about access modifiers.

To create a child class, \texttt{Dog}, we simply create a class an extend from \texttt{Animal} using the \texttt{extend} keyword (line 1 in listing \ref{lst:dog}:

\begin{lstlisting}[language=Java, label=lst:dog]
public class Dog extends Animal {
 // ...
}
\end{lstlisting}

Cats and dogs also have unique features.  Cats have fur balls and dogs wag their tails. We can add fields and methods to the child class:
\begin{lstlisting}[language=Java]
// Dog: 
private int wagCount;

public int getWagCount() {
 return wagCount;
}

public void setWagCount(int wagCount) {
 this.wagCount = wagCount;
}

// Cat:
private int furballCount;

public int getFurballCount() {
 return furballCount;
}

public void setFurballCount(int furballCount) {
 this.furballCount = furballCount;
}
\end{lstlisting}
Appendix J on page \pageref{App:AppendixJ} has the the source of the classes \texttt{Cat.java} and \texttt{Dog.java}.
We can now create \texttt{Cat}s and \texttt{Dog}s:
\begin{lstlisting}[language=Java]

Animal pluto = new Dog("Pluto", Animal.Sociability.VERY_SOCIAL, 3);
Animal sheba = new Cat("Sheba", 2);

// We can also create a Dog object
Dog pepper = new Dog("Pepper", Animal.Sociability.VERY_SOCIAL, 99);

\end{lstlisting}

The difference between \texttt{pluto} and \texttt{pepper} is that the \texttt{pluto} reference is of type \texttt{Animal}, and the \texttt{pepper} reference is of type \texttt{Dog}. There usually isn't a good reason to use the specific reference (\texttt{pepper}). It conceptually goes against the purpose of inheritance.

If the super class has any public or protected methods, the child classes can call them. For example, if \texttt{Animal} has a method walk:
\begin{lstlisting}[language=Java]
protected String walk() {
 String steps = this.getName() + " walks: ";
 for (int i = 0; i < numberOfLegs; i++) {
  steps += " step";
 }
 return steps;
}
\end{lstlisting}

It simply writes "step" once for each leg.

We can call it from the \texttt{Dog} and \texttt{Cat} classes:

\begin{lstlisting}[language=Java]
// Inside a method in Cat:
String whatDoesTheCatSay = "Meouw ";
whatDoesTheCatSay += super.walk();
\end{lstlisting}

The \texttt{super} reference is optional (line 2). We could as well call the \texttt{walk()} method without it. After all a \texttt{Cat} "is-an" \texttt{Animal}. Using the \texttt{super} keyword makes the code clearer to read, though.

Executing the above code produces:
\begin{lstlisting}
Bark Pluto walks:  step step step step
Meouw Sheba walks:  step step step step
\end{lstlisting}


Since the child classes inherit the super classes public and protected members, we can call the \texttt{walk()} method from a reference:

\begin{lstlisting}[language=Java]
Animal spot = new Dog("Spot");

// Using a Dog reference works as well:
// Dog spot = new Dog("Spot"); 

spot.walk();
\end{lstlisting}

The output is:

\begin{lstlisting}
Spot walks:  step step step step
\end{lstlisting}

There is no way to force \texttt{Dog} call the \texttt{walk()} method. It is available for the child classes but they don't have to use it. If there is something that we want to enforce, there is a better way.

In \texttt{Animal} we can specify an abstract method \texttt{say()}. This way we force an \texttt{Animal} to have that functionality, but we force the child classes to specify what it means. 
\begin{lstlisting}[language=Java]
public abstract String say();
\end{lstlisting}

The method signature includes the \texttt{abstract} keyword, and there is no method body. The body will be implemented in each child class. In order to make a method \texttt{abstract} we also must mark the class \texttt{abstract} because there is no sense to instantiate an object that doesn't have a method implementation.

Although \texttt{Animal}s say things, they say different things. Dogs bark and cats meow. In the child classes we override the \texttt{say} method and implement it appropriately for each child class:
\begin{lstlisting}[language=Java]
// Dog:
@Override
public String say() {
 String whatDoesTheDogSay = "Bark";
 for (int i = 0; i < wagCount; i++) {
  whatDoesTheDogSay += " (wag) ";
 }
 whatDoesTheDogSay += super.walk();
 return whatDoesTheDogSay;
}
  
// Cat:
@Override
public String say() {
String whatDoesTheCatSay = "Meouw";
 for (int i = 0; i < furballCount; i++) {
  whatDoesTheCatSay += " (cough) ";
 }
 whatDoesTheCatSay += super.walk();
 return whatDoesTheCatSay;
}
\end{lstlisting}
Providing an alternate implementation of a method in the child class is called overriding. The overriding method has to have the same signature (return type and parameter list) as the method in the parent class. Java provides an annotation for marking a method overridden, namely \texttt{@Override}. If we use it and our method doesn't match the parent method, we will see an error. The annotation is useful for keeping us in check when overriding methods.

The \texttt{Dog} says "Bark", then wags the tail as many times as indicated by the wagCount, then walks (by calling the \texttt{super.walk()} method).

The \texttt{Cat} says "Meouw", coughs some furballs, and also walks.

We can create an \texttt{Animal} of type \texttt{Dog} and call the \texttt{say} method:

\begin{lstlisting}[language=Java]
Animal pepper = new Dog("Pepper", Animal.Sociability.VERY_SOCIAL, 99, new Tail(true, 1));
System.out.println(pepper.say());

Animal sheba = new Cat("Sheba", 2, new Tail(false, 15));
System.out.println(sheba.say());
\end{lstlisting}


When we call the \texttt{say} method we have the following output:
\begin{lstlisting}
Bark (wag)  (wag) ... (wag)
  Pepper walks:  step step step step

Meouw (cough)  (cough)
  Sheba walks:  step step step step
\end{lstlisting}

By forcing the child classes to implement a method, the super class can call it. Even when the \texttt{say()} method is not implemented in the \texttt{Animal} class, we can call it on an \texttt{Animal} reference. The JVM will find the appropriate implementation and execute that. This is called "virtual method invocation".

One of the benefits of inheritance is that we can treat Cat and Dog objects the same way, as instances of the Animal class. With that, we can, for example, store Cats and Dogs in the same collection and iterate through them.
\begin{lstlisting}[language=Java]
List<Animal> pets = new ArrayList<Animal>();
pets.add(spot);
pets.add(sheba);

System.out.println("\n\nAnimals in a list");
for(Animal animal : pets){
 System.out.println(animal);
}
\end{lstlisting}

The output looks like:
\begin{lstlisting}
Animal:{  
 name='Spot',
 sociability=VERY_SOCIAL,
 numberOfLegs=4,
 tail=Tail      {  
  docked=false,
  length=8
 }
}
Bark (wag)  (wag)  (wag)  (wag)  (wag) 
Spot walks:step step step step,

Animal:{  
 name='Sheba',
 sociability=VERY_UNSOCIAL,
 numberOfLegs=4,
 tail=Tail      {  
  docked=false,
  length=15
 }
}
Meouw (cough)  (cough)
Sheba walks:step step step step
\end{lstlisting}

See Appendix J on page \pageref{App:AppendixJ} for full source code of the \texttt{InheritanceCompositionExample.java} class.

\subsubsection{Composition}
\begin{figure}[!h]\centering % Using \begin{figure*} makes the figure take up the entire width of the page
\includegraphics[width=0.9\linewidth, frame]{images/composition}
\caption{Composition ("Has-A")}
\label{fig:composition}
\end{figure}

A composition is another way to share features between classes. The easiest way to think of it is a "has-a" relationship. For example, a \texttt{Cat} and a \texttt{Dog} "have-a" \texttt{Tail}.

First we create a \texttt{Tail} class.
\begin{lstlisting}[language=Java]
public class Tail {
 private boolean 	;
 private int length;

 public Tail(boolean docked, int length) {
  this.docked = docked;
  this.length = length;
 }
}
\end{lstlisting}

We can make the \texttt{Animal} class "have-a" tail (although not all animals have tails, we simplify it for the sake of the example). Because \texttt{Dog} "is-an" \texttt{Animal}, it also "has-a" \texttt{Tail}. The same for the \texttt{Cat}.

Appendix J on page \pageref{App:AppendixJ} has the source code for the  \texttt{Tail.java} class.

We just create a field in the \texttt{Animal} class for the tail.
\begin{lstlisting}[language=Java]
private Tail tail;
\end{lstlisting}

Now we can access it from \texttt{Dog} or \texttt{Cat}, as well as from \texttt{Animal}:
\begin{lstlisting}[language=Java]
Tail tail = new Tail(false, 10);
pluto.setTail(tail);
System.out.println(pluto.getTail().isDocked());
\end{lstlisting}

The above code will output \texttt{false} because \texttt{pluto}'s tail is not docked.

\subsubsection{Static Methods}
There is yet another way to provide common functionality among several classes. We can declare a method or field \texttt{static}d:

\begin{lstlisting}[language=Java]
public static String adopt(){
 return "Yay! Adopted Animal!";
}
\end{lstlisting}

The \texttt{static} keyword makes the class, field, or method special. We don't need to instantiate an object to access \texttt{static} members of a class. They are class-level parameters and fields:
\begin{lstlisting}[language=Java]
System.out.println(Animal.adopt());

// displays
Yay! Adopted Animal!
\end{lstlisting}
We can access the \texttt{static} \texttt{adopt()} method from the class, without using the \texttt{new} keyword to create an object.

A class might want to provide \texttt{static} fields, for example constants. In such case they should be also marked \texttt{final}, to avoid modification.

Also anything that we might want to share between objects of the same class should be marked \texttt{static}. For example, we might (for some strange reason) want to keep a count of instances of a class. If we declare a \texttt{static} field, all instances of the class will access exactly the same copy of that field. If one modifies it, it will be also show modified for the other instances (unless it is marked \texttt{final}, in which case it can't be modified).

\subsubsection{Inner Classes}
It is worth mentioning one more way of sharing code between classes. In Java we can have inner classes, or classes inside of other classes. They have access to all the fields and methods of the outer class, including the private members.

An inner class can be instantiated from outside the outer class, if its access modifiers are set properly. But, the instantiation has to happen through an object of the outer object:
\begin{lstlisting}[language=Java]
OuterClass outer = new OuterClass();
OuterClass.InnerClass inner = outer.new InnerClass();
\end{lstlisting}

The inner object now has access to everything in the outer class. We could create two or more inner classes this way, and they all have access to common methods and fields in the outer class. 

It is not an "is-a", nor a "has-a" relationship. In some ways it breaks encapsulation and doesn't really look like object-oriented programming at all.
\section{Hypermedia}

Hypermedia is this and that

\subsection{Some Topic}

something this and that

\section{Web Technologies}

Something here
\input{section-04-subsection-01-some-topic.tex}

\section{Enunciate}

Blah

\input{section-05-subsection-01-some-topic.tex}
\phantomsection
\begin{thebibliography}{21}

\bibitem{spec}
James Gosling, Bill Joy, Guy Steele, Gilad Bracha, Alex Buckley. \textit{The Java® Language Specification.} The Java® Language Specification. Oracle America, Inc., 13 Feb. 2015. Web. 12 Dec. 2016. <\url{http://docs.oracle.com/javase/specs/jls/se8/html/index.html}>.

\qrcode[height=.5in]{http://docs.oracle.com/javase/specs/jls/se8/html/index.html}




\end{thebibliography}

\onecolumn
\phantomsection
\section*{Appendix} % The \section*{} command stops section numbering

\addcontentsline{toc}{section}{Appendix} % Adds this section to the table of contents

\appendix
\subsection*{Appendix A -- Something} \label{App:AppendixA}
%\addcontentsline{toc}{subsection}{Appendix A -- Garbage Collection} % Adds this section to the table of contents

\subsubsection*{org.familysearch.viitanenm.gc.GarbageCollectionExample.java}
\noindent\begin{minipage}{.6in}
  \qrcode[height=.5in]{https://raw.githubusercontent.com/mviitanen/ati-java-apprentice/master/code/1-1-garbage-collection/src/org/familysearch/viitanenm/gc/GarbageCollectionExample.java}
\end{minipage}
%  \hspace{0.3cm}% adjust for horizontal spacing
\begin{minipage}{6in}
  \url{https://raw.githubusercontent.com/mviitanen/ati-java-apprentice/master/code/1-1-garbage-collection/src/org/familysearch/viitanenm/gc/GarbageCollectionExample.java}
\end{minipage}

\vspace{1em}
\subsubsection*{org.familysearch.viitanenm.gc.DataSizePrinter.java}
\noindent\begin{minipage}{.6in}
      \qrcode[height=.5in]{https://raw.githubusercontent.com/mviitanen/ati-java-apprentice/master/code/1-1-garbage-collection/src/org/familysearch/viitanenm/gc/DataSizePrinter.java}
    \end{minipage}
 % \hspace{0.3cm}% adjust for horizontal spacing
    \begin{minipage}{6in}    
      \url{https://raw.githubusercontent.com/mviitanen/ati-java-apprentice/master/code/1-1-garbage-collection/src/org/familysearch/viitanenm/gc/DataSizePrinter.java}
    \end{minipage}

%----------------------------------------------------------------------------------------

\end{document}